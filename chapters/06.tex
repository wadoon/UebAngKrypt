\section{Digitale Signatur und Zertifikate}

\subsection{Keine Signatur mit dem Rucksack}

\subsection{RSA-Signatur}
	\[n = 3 * 5 = 15, e = 3, M = 7\]
	
   \[ d = 3 \imp 3 \cdot 3 \equiv 1 \mod 8 \]	
	
\subsubsection{Berechnen Sie die digitale Unterschrift nach dem RSA-Verfahren.}

\begin{align}
	D_{SK}(M) = 7^{3} \mod 15 = 13
\end{align}

\subsubsection{Was übertragt der Sender zum Empfänger, wenn er die Nachricht M signiert übertragen will?}
	
	\[ (M,D_{SK}) = (7,13) \]

\subsubsection{Verifizieren Sie die Unterschrift.}

\[ E_{K} = 13^{3} \mod 15 = 7 = M \]


\subsection{Elgamal-Signatur}
\[ p = 467, g = 2, x = 127, y = 2127 = 132 mod 467\]
Die Nachricht $M = 100$ soll unterschrieben werden. Es wird $k = 213$ verwendet.
\subsubsection{Berechnen Sie die digitale Unterschrift nach dem Elgamal-Verfahren.}

\[ a = g^k \mod p = 2^213 \mod 467 = 29 \]

\begin{align}
	                 M &= (xa + kb) \mod (p-1) \\
                     M &= ((127\cdot 29 \mod (467-1)) + (213 \cdot b \mod (467-1))) \mod (467-1) \\
               M - 421 &=  213 \cdot b \mod (467-1) \\
	\dfrac{M-421}{213} &= b \mod 466 \\
	(100 - 421) \cdot 213^{-1} &\imp b = -321\cdot 213^{-1} \mod 466 \\
							   &\imp b = -321 \cdot -35 \mod 466 = 51	
\end{align}

\subsubsection{Was übertragt der Sender zum Empfänger, wenn er die Nachricht M signiert
übertragen will?}

	\[ (M, (a,b)) = (100, (29,51)) \]

\subsubsection{Verifizieren Sie die Unterschrift.}

\begin{align}
	g^M     &\equiv y^a a^b \mod p 					\\
	2^{100} &\equiv 132^{29} \cdot 29^{51} \mod 467	\\
	189     &\equiv 189		
\end{align}

\subsubsection{Länge der Passphrase für digitale Signatur}

Sie haben einen 1024 Bit (2048 Bit) RSA-Schlüssel. Wie lang sollte die Passphrase
zum Schutz des auf Ihrer Festplatte gespeicherten Schlüssels mindestens sein?
Hinweis: Nehmen Sie an, dass sie Sicherheit eines 1024 Bit RSA-Schlüssels einem
128 Bit symmetrischen Schlüssels entspricht.

\textit{Angenommen:} Passwort $w$ $\in \Sigma^*$ besteht aus den Terminalsymbolen aus ASCII.
\[|\Sigma| = 256-32 = 224\]

\textit{Frage:} Welche Länge muss mein Passwort haben, damit $2^128$ Möglichkeiten habe.

	\[ | \{ w \in \Sigma^* | |w| = x \} | = 2^{128} \]
\begin{align}
		|\Sigma|^|w| &= 2^{128} \\ 
		224^x        &= 2^{128} \\
		|w|\log_2(224) &= 128 \\
		|w|            &=  16,395
\end{align}	

für $|\Sigma| = 64 \imp |w| = 21,3 $


\subsection{GPG}
\subsection{Signierung eines Java-Applets}
\subsection{PDF-Signatur}

